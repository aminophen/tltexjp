%<?xml version="1.0" encoding="iso-8859-1"?>
%% -*-LaTeX-*-
%%
%% EuroTeX 2003 - Back to typography
%% 14th European TeX conference, ENST Bretagne, Brest, June 24-27, 2003.
%% http://omega.enstb.org/eurotex2003/
%%
%% $Id: freefont.tex,v 1.2 2005-11-16 15:02:06 peterlin Exp $
\documentclass[english]{eurotex2003}
\usepackage{url}
\newfont{\cyr}{wncyr10}

% XXX: Is `seriffed' correct? (yes; so is just "serif"; changed one
% occurrence to "serif" to improve line breaks --karl)
\advance\aboveabstractskip by -9pt
\advance\interabstractskip by -3pt
\advance\interlinepenalty by -200
\hbadness=3900
\hyphenation{Pfa-Edit}
\begin{document}%\setcounter{page}{260}
\frenchspacing

\title{The Free UCS Outline Fonts Project\Dash{}an Attempt to Create a Global
Font}
\author{Primo\v{z} Peterlin}
\address{University of Ljubljana\\
Faculty of Medicine, Institute of Biophysics\\
Lipi\v{c}eva 2, SI-1000 Ljubljana, Slovenia}
\netaddress{primoz.peterlin@biofiz.mf.uni-lj.si}
\personalURL{http://biofiz.mf.uni-lj.si/~peterlin/}

\begin{abstract}
In February 2002, the Free \acro{UCS} (Universal Character Set) Outline Fonts
project (\url{http://savannah.gnu.org/projects/freefont/}) was
started. Exercising the open-source approach, its aim is to provide a
set of free Times-, Helvetica- and Courier-lookalikes available in the
OpenType format, and progressively cover the complete \acro{ISO}
10646/Unicode range. In this stage of the project, we focus mainly on
two areas: collecting existing fonts that are both typographically and
license-wise (i.e., \acro{GNU GPL}) compatible and can be included to cover
certain parts of the character set, and patching up smaller areas that
are not yet covered. Planned future activities involve typographic
refinement, extending kerning information beyond the basic Latin
area, including TrueType hinting instructions, and facilitating the
usage of fonts with various applications, including the \TeX/\OMEGA{}
typesetting system.
\end{abstract}

\begin{resume}
\hbadness=7400
Le projet de \emph{fontes vectorielles libres pour \acro{UCS}}
(\url{http://savannah.gnu.org/projects/freefont/}) a d�marr� en f�vrier
2002. \smash{�} travers l'\emph{Open Source}, le but de ce projet est de fournir
un ensemble de clones libres de Times, Helvetica et Courier, disponibles
dans le format OpenType, et couvrant progressivement l'ensemble des
caract�res d'\acro{ISO} 10646/Unicode. \smash{�} ce stage du projet nous nous
concentrons sur deux domaines : la collection de fontes existantes qui
soient compatibles avec notre projet, aussi bien du point de vue
typographique que du point de vue de la licence (\acro{GNU GPL}), qui puissent
�tre int�gr�es pour couvrir certaines parties de l'ensemble de
caract�res ; et d'autre part, faire des ajouts de petites r�gions qui
n'ont pas encore �t� couvertes. Nos activit�s futures pr�voient le
perfectionnement typographique, le cr�nage au-del� de l'alphabet latin,
l'incorporation de \emph{hints} TrueType, et le support de plusieurs
applications, dont aussi les syst�mes \TeX{} et \OMEGA.
\end{resume}

\maketitle

\section{Introduction}

The aim of the Free \acro{UCS} Outline Font project is to provide a
standardised set of glyphs which make a harmonised design despite
including glyphs from different scripts (Latin, Cyrillic,
Greek, Armenian, etc.). It is clear that this requires compromises at
the cost of typographic finesse (see
\cite{Haralambous:UnicodeTypography} for a discussion of typographic
compromises regarding the Greek alphabet), yet the end result must
look acceptable for general use, including electronic mail, world-wide
web, and text editors.

% XXX: Either mention Kudryashov's name transcribed to the Latin script
%      directly in parentheses, or provide a reference which shows a
%      transcription.
While this is clearly not the first attempt to create a typeface
covering glyphs beyond Latin, previous attempts are not as numerous as
one might imagine. Bigelow \cite{Bigelow:1993} quotes \emph{Romulus} by
Jan van Krimpen from 1931 as one of the first examples. Another
example is {\cyr Niko\-lai0 Nikolaevich Kudryashov} [Nikola\u{\i}
Nikolaevich Kud\-rya\-shov]'s
\emph{Encyclopaedia} ({\cyr Kud\-rya\-shov\-ska\-ya
enciklo\-pedi\-cheskaya} [Kud\-rya\-shov\-ska\-ya entsiklopedicheskaya])
family (1960--1974), originally designed for the third edition of the
Great Soviet Encyclopaedia which contains thousands of glyphs,
including Cyrillic, Latin, and Greek letters in serif and sans-serif
styles, as well as other special signs and symbols. Most popular
typefaces such as Stanley Morison's Times Roman, Max Miedinger's
Helvetica, Adrian Frutiger's Univers and Eric Gill's Gill Sans were
originally designed for the Latin alphabet and later, as they gained
popularity, extended to Greek and Cyrillic alphabets.
% See http://typo.mania.ru/FACES/b-kudrsh.htm for info on Kudryashov.

% XXX: It should be made clear that `typeface' doesn't necessarily mean
%      a single font file which in most cases is far too big to process
%      efficiently.
A typeface covering the whole \acro{ISO} 10646/Unicode range has to be
designed flexibly enough to allow addition of scripts not belonging to
the Western typographic tradition. The task is not easy, and it is not
surprising that so far there are very few aesthetically satisfying
typographic solutions. Among those one has to mention Bigelow and
Holmes's \emph{Lucida Sans Unicode} \cite{Bigelow:1993} and
Haralambous and Plaice's {\OMEGA} \cite{Haralambous:1994}. Other
solutions like \emph{Arial Unicode} \cite{ArialUnicode} and James
Kass's compendium \emph{Code2000} \cite{Code2000} are too varied in
style to form a unified typeface.

\section{Design issues}

\subsection{Historical and cultural context.}

It is clear that a typeface aiming to cover ``all the scripts of the
world'' can not rely on historical styles (e.g., Renaissance, Baroque,
etc.\@) known from Western typography, as most of the world has not
experienced these periods in the evolution of typography. Trying to
extend them to non-European scripts\footnote{Even though historical
styles differ throughout Europe, Europe is nevertheless treated here
as a historical and cultural unity when contrasted with the rest of
the world.} is as inappropriate as, say, trying to design a Latin
alphabet in Kufi style.

% XXX: Is stone-carvers correct?  [sure --karl]
Even the apparently logical division into seriffed and sans-serif
typefaces is eurocentric. To see why, one only has to think of the
origin of Latin capital letters. Serifs are an invention of Roman
stone-carvers\Dash{}a smaller stroke perpendicular to the main
stroke was added to provide a uniform smooth finishing of the stroke.
Here, the visual effect followed the available technology. The
modulated stroke\Dash{}another ancient Roman typographic invention\Dash{}is
an opposite example, where the technology followed the visual effect
\cite{Jean}. The need to strengthen vertical strokes arose once the
Romans started to erect monuments of monumental proportions, where the
inscriptions were no longer in the eye-height. While the rain took
away the paint from the vertical strokes, dirt and dust accumulated in
the horizontal strokes, which thus appeared optically
heavier. \emph{Physical} broadening of vertical strokes was introduced
as a compensation, in order for the horizontal and vertical strokes to
look \emph{optically} equivalent. The Roman technological innovations
survived for 2\,000 years in European typography.
However, in countries where
the letters were painted with brush onto silk or carved with a needle
onto a palm leaf rather than carved into the stone with mallet and
chisel, such typographic development never happened.

Similar concerns about eurocentrism are valid for differing
the upright and the italic forms, or for that matter, even
between \emph{majuscules} and \emph{minuscules}, capital and small
letters. Even differing between upright and slanted forms is
questionable, as slanted forms make no sense in, say, most native Asian
scripts.

The least questionable seems to be the differences based on the weight
of the typefaces, which appears to be almost universal. The only
exception known to the author is Ethiopic,\footnote{Daniel Yacob, personal
communication.} where the words were traditionally
emphasised by printing them in another colour (red in religious texts,
blue in imperial decrees) or by underlining them or enclosing them in
ovals.

\subsection{Legibility.}

A typeface which includes non-Latin scripts offers an opportunity for
exploring the ``universal'' parameters of legibility. While there has
been a wealth of publication on legibility centred on Latin script
\cite{Tinker,Zachrisson,deLange,Lund}, exploring the factors like weight,
serif vs. sans-serif faces, x-height, capitalisation etc., it is clear
that such differences are minor compared to the differences in shape
between Latin and, say, Hebrew, Arabic, Devanagari or Tamil, which
nevertheless provide roughly the same level of legibility. Lacking
comparative cross-script studies, the experiments in legibility remain
in the realm of the typographer's intuition. On the brighter side, making
multi-script typefaces available, albeit not perfect, is a step
towards the world where such cross-cultural studies will be easier to
achieve.

\section{Methodology}

The basic idea behind the free \acro{UCS} outline fonts project was to
collect various available free outline fonts, covering single national
scripts, and to compile them into a large font using the \acro{ISO} 10646/Unicode
coded character set \cite{Unicode3}, taking into consideration
typographic and legal compatibility, and filling in the missing areas
on the way. The whole development was planned to be carried in the
open-source manner, with many developers using a central
repository.

The general requirement for technical realisation was that typefaces
need to be available as scalable vector fonts. The actual technical
realisation (Post\-Script Type~1 \cite{Adobe:Type1} uses cubic B\'ezier
splines, while TrueType \cite{Apple:TrueType} uses quadratic ones) was
considered secondary, because at least in principle it is possible
to transform the fonts from one form to another. In reality, though,
no known transformations is completely lossless\Dash{}kerning and hints
are usually the most volatile.

\subsection{Licenses of used sources.} We anticipated that our result
will contain glyphs from many different sources. Thus, special
attention was given to the license under which a font is released. The
license we looked for should allow redistribution, modification and
distribution of modified font files. Many free and open-source
licenses fulfil this requirement. As the \acro{URW++} core PostScript fonts,
the {\OMEGA}-Serif and some other major sources were released under the
\acro{GNU} General Public License \cite{GNU:GPL}, we adopted it for our
project as well. The license itself is suited for programs rather than
typefaces and its application to fonts may be legally dubious, even
though, say, PostScript Type~1 fonts are perfectly legal programs
written in PostScript. We were not able to find any license in the
same spirit pertaining specifically to fonts, though. It may be worth
seeing the final license of the Bitstream Vera fonts, which were
announced in January 2003 to be soon available under a license in the
open-source spirit.

We thus limited our search to fonts which were not only
typographically compatible, but also specifically released under the \acro{GNU
GPL}. In some cases, where the fonts were released under less clearly
defined terms (``free'', ``public domain'' etc.), we contacted the
authors and asked for permission to use their work under the terms of the \acro{GNU
GPL}. Most authors agreed, and while none disagreed, some of the emails
remained unanswered. In such cases we were unable to confirm whether
the recipient received our email at all. These fonts still wait to be
included in the free \acro{UCS} outline font collection.

\subsection{Choice of typefaces.} Even though Bigelow and Holmes claim
that their primary reason for choosing a sans-serif font is because it
carries least historical and cultural associations
\cite[p. 1003]{Bigelow:1993}, and despite the expressed concerns about
eurocentrism, we decided to develop in parallel three different
families, modelled after Times Roman, Helvetica and Courier, and
specifically derived from \acro{URW++} typefaces Nimbus Roman No.~9, Nimbus
Sans and Nimbus Mono. Reflecting the nature of the project, we dubbed
them as Free Serif, Free Sans and Free Mono.

Aside from covering three different letter\-forms\Dash{}one monospaced and
two proportional, one with modulated strokes and another with
unmodulated strokes\Dash{}the primary reason for choosing these three
typefaces was their ubiquity. Since they have been extremely popular for
many decades, and present in the desktop publishing world-wide for over
twenty years, they inspired numerous local designs around the world,
where non-Latin glyphs were designed specifically to blend with one of
the above-named typefaces. While we realise that many of these designs introduce
typographic practice alien to a traditional local design, we
also recognise that the designs done by native speakers
reflect the local typographic knowledge and its evolving typographic
rules.

\subsection{Tools.} While we paid special attention to the licensing of
the used sources, no particular attention was given to the licensing
issues of the tools used, as the font tools generally don't imply the
licenses under which the fonts are released. Still, we ended up using
exclusively open-sourced tools. The choice was clearly influenced by
the fact that most participants of the project use Linux, and while
none of the numerous popular font editors from other platforms have been
ported to this platform, we have some excellent free tools available.
\begin{description}
\item[PfaEdit] Without George Williams's excellent font editor
PfaEdit,\footnote{Since renamed to FontForge:\\\hspace*{2.5em}\url{http://fontforge.sourceforge.net/}} this project would not have
been possible at all. PfaEdit is a visual font editor that allows
copying and modifying glyphs and most other things expected from a
font editor, reads and writes most popular outline font formats, and
has proved to be stable enough to allow working on large glyph
sets. Even though the program was mature enough to be used as
a tool back in late 2001, the pace of its development has not
diminished over the past year, and its author is very responsive to
bug reports.

Choosing PfaEdit as our main tool also influenced the decision of the
native font format used in the free \acro{UCS} outline fonts project. Since
we use a \acro{CVS} repository for the bookkeeping of font additions and
modifications, we were looking for a format suitable to differential
changes. A format was considered suitable if local changes in typeface
design remained localised in the font file rather than propagating
themselves across the file. This requirement ruled out compiled binary
formats, as well as PostScript Type~1 \texttt{eexec} encoding.
PfaEdit's native format \acro{SFD}, similar to Adobe's \acro{BDF} format in its
structure, proved to be a suitable choice. The fonts are thus archived
in the \acro{SFD} format, and PfaEdit is used to create fonts in other
outline formats.

PfaEdit's only major drawback may be its weak support for TrueType
instructions or ``hints''. Since TrueType is not the native format
used by the free \acro{UCS} outline font project, we are looking either for
automated generation of TrueType hints, or for separate files with
TrueType instructions which could be compiled together with the glyph
outlines in the \acro{SFD} format to produce a hinted TrueType font. Any
solution requiring manual modification of binary TrueType fonts is not
acceptable, as TrueType fonts are created automatically each time from
the \acro{SFD} sources. While PfaEdit will probably eventually produce
acceptable TrueType hints in an automated way,
other possibilities have to be investigated in the meantime.

\item[TrueType tools] \ Rogier van Dalen's TrueType tools\footnote{\url{http://home.kabelfoon.nl/~slam/fonts/}} might be the proper
solution for the TrueType hinting problem. Van Dalen's approach uses a
separate file with TrueType instructions written in a high-level
language not unlike~C, which can be compiled together with an unhinted
TrueType font file to produce a hinted TrueType font file. If this
approach is adopted, each TrueType font file will be automatically
produced from two source files: the \acro{SFD} file containing the glyph
outlines and the \acro{TTI} file containing the TrueType instructions for
hinting. At the moment of this writing, the main concern is the
estimated amount of work needed for producing TrueType hinting
instructions in a manual way.

% XXX: does the word `ogling' exist? (it does, but we don't want to ogle
% TTX; changed to "considering" --karl)
\item[\acro{TTX}] While the \acro{SFD} format fits present needs just fine, the
future might require distributing font sources in some other open
format. For this purpose we are considering \acro{TTX}, the TrueType to \acro{XML}
converter.\footnote{\url{http://www.letterror.com/code/ttx/}} Given the pace
at which PfaEdit currently evolves, it may well be that PfaEdit itself will
support \acro{XML} before we will actually feel the need for~it.

\end{description}

\subsection{Towards \acro{ISO} 10646/Unicode compliance.}

As of version 3.2, Unicode defines 95\,156 encoded characters. Taking
into account three families (Free Serif, Free Sans and Free Mono), two
weights (normal and bold) and two shapes (regular and
italic/ob\-lique), this means designing almost 1.2~million glyphs for
the free \acro{UCS} font project. Being a volunteer project, free \acro{UCS} outline
fonts project grows mainly in a non-systematic way\Dash{}when a suitable
set of glyphs is found or donated, we add it to the font. Often, a
wider community can benefit from scratching one's own itch, e.g. creating
a set of \acro{APL} glyphs \cite{Chastney:1999,Chastney:2001}.

Still, we felt the need to add the glyphs in a more systematic way in
parallel with the spontaneous growth of the project. For that purpose,
we use the multilingual European subsets as defined by Co\-mi\-t\'e
Euro\-p\'en de Normalisation \cite{CWA13873} as a guideline:
\begin{itemize}
\item \acro{MES}-1, a Latin repertoire based on \acro{ISO/IEC}\\ 6937:1994 (335
characters)
\item \acro{MES}-2, a Latin, Cyrillic, and Greek repertoire based on \acro{ENV}
1973:1996 (1062 characters)
\item \acro{MES}-3, a repertoire needed to write all the languages of Europe
and transliterate between them. \acro{MES}-3A is a script-based, non-fixed
collection, while \acro{MES}-3B is a fixed subset of 2819 characters.
\end{itemize}

\subsection{Character-to-glyph translation.}

As the \acro{ISO} 10646/Unicode standard encodes \emph{characters} rather than
\emph{glyphs}, it is clear that for acceptable rendering of many
langua\-ges more glyphs than those corresponding to Unicode characters
are needed. This is particularly important for rendering Indic
languages and Syriac which contain numerous ligatures not encoded in
the \acro{ISO} 10646/Unicode standard.

During the 1990's, several initiatives for creating ``smart fonts'' were
started, i.e., fonts being not only containers of glyphs but
also incorporating some logical rules for glyph substitution, glyph
position, contextual rendering, etc. Among those were Apple Advanced
Typography \cite{Apple:AAT}, \acro{SIL} Graphite \cite{SIL:Graphite} and
Adobe/Microsoft OpenType \cite{Microsoft:OpenType}. From these three,
OpenType seems to be grabbing the largest market share.

\section{Results}

The project was conceived in December 2001 and was approved for
hosting on the Savannah Web site\footnote{\url{http://savannah.gnu.org/projects/freefont}} in February
2002. As of March 2003, a total of 17\,794 characters are encoded. We
are just a couple of dozen glyphs short of \acro{MES}-1 compliance in Free
Serif and Free Sans, while Free Mono is already \acro{MES}-1 compliant, and
approximately 3\,500 glyphs short of \acro{MES}-2 compliance.

{%\tolerance=7000

\subsection{Kerning.}
Aside from the character set compliance and the already mentioned
TrueType hinting problem, there remain some other tasks. One
of them is kerning, which is currently present only in the Latin
portions of the font. Kerning non-Latin scripts requires typographic
knowledge which we currently do not possess. We believe that the open nature of
the project will eventually attract somebody with the necessary
expertise who is willing to contribute.

\section{Discussion}

The open-source development model has not been previously tested for font
development. While it may be arguable whether such a development model
can be applied at all to creation of works of art, it is also worth
noting that most contributors have a technical and scientific
background, and probably none of them perceives him- or herself as an
artist.

While there is no doubt that a skilled typographer would be able to
design a multi-script font vastly exceeding what we have done in
typographic beauty and consistency, we must not overlook the fact that
designing 1\,200\,000 glyphs is probably beyond the capability of a
single person, even if we neglect the financial part of such an
endeavour.

% XXX: The pdf file I received hyphenates Sz-ab\'o!  How does
%      this come?  A quick test with \showhyphens shows that hyphen.us
%      doesn't do this.  Ts, ts, ts.
%      (no hyphenation at current sizes anyway --karl)
\subsection{Free \acro{UCS} outline fonts project and \TeX.}
So far, the Free \acro{UCS} outline fonts project has taken from the \TeX\
community more than it has repaid. Thanks to P\'e\-ter Sza\-b\'o and his
{\TeX}trace program (\url{http://www.inf.bme.hu/~pts/textrace/})
\cite{Szabo:TeXtrace}, it is easy to transform Metafonts into
PostScript Type~1 fonts which can be edited by PfaEdit. Karel
Pi\v{s}ka demonstrated the technique for Indian Metafonts during the \acro{TUG}
2002 conference in Thiruvananthapuram, India.

Even though the free \acro{UCS} outline fonts were designed primarily for
screen use, we can certainly raise the question of using the fonts for
typesetting with \TeX\ or {\OMEGA}. As of March 2003, this would
require extensive work, such as splitting the fonts into smaller
fonts containing no more than 256 glyphs. However, the current
development \cite{Bella-Mehta} promises that there might be a more
direct way of using OpenType fonts with {\OMEGA} in the future.

\section{Acknowledgments}

I would like to thank the following authors who contributed to the
project by allowing their work to be included in the free UCS outline
fonts project. In alphabetical order: Berhanu Beyene, Daniel
Shurovich Chirkov, Prasad A. Chodavarapu, Vyacheslav Dikonov, \acro{DMS}
Electronics, Valek Filippov, Shaheed R. Haque, Yannis Haralambous,
Angelo Haritsis, Jeroen Hellingman, Maxim Iorsh, Mohamed Ishan,
Manfred Kudlek, Harsh Kumar, Sushant Kumar Dash, Olaf Kummer, Noah
Levitt, Jochen Metzinger, Anshuman Pandey, Hardip Singh Pannu, Tho\-mas
Ridgeway, Young U. Ryu, Virach Sornlertlamvanich, M.S. Sridhar, Sam
Stepanyan, \acro{URW++} Design \& De\-ve\-lop\-ment GmbH, Frans Velthuis, and the Wadalab
Kanji Committee.

}

\hbadness=10000
\bibliographystyle{plain}
% XXX: There is a missing `~' as in `Type~1' in the reference to Szabo.
% that break doesn't happen now --karl
\bibliography{freefont}
\end{document}
