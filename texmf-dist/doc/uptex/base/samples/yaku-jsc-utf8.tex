\makeatletter
\def\@default{default}
\def\jsbook{jsbook}
\ifx\class\@undefined
 \def\class{jsarticle}
\fi
\ifx\option\@undefined
 \def\option{default}
\fi
\ifx\option\@default
 \documentclass[uplatex]{\class}
\else
 \documentclass[uplatex,\option]{\class}
\fi
\makeatother

\DeclareFontFamily{JY2}{umin}{}
\DeclareFontFamily{JY2}{ujis}{}
\DeclareFontFamily{JY2}{upjisr}{}
\DeclareFontShape{JY2}{umin}{m}{n}{<->s*[0.961]umin10}{}
\DeclareFontShape{JY2}{ujis}{m}{n}{<->s*[0.961]ujis}{}
\DeclareFontShape{JY2}{upjisr}{m}{n}{<->s*[0.92469]upjisr-h}{}
\DeclareRobustCommand\umin{\kanjifamily{umin}\selectfont}
\DeclareRobustCommand\ujis{\kanjifamily{ujis}\selectfont}
\DeclareRobustCommand\upjisr{\kanjifamily{upjisr}\selectfont}

\begin{document}
option: uplatex and \option
\ifx\class\jsbook
\chapter{新ドキュメントクラス}
\fi
\section{拗音、句読点など}
\xdef\testtext{%
カムチャッカ.
ちょっと、待って。きっと,ショック.~~
‘回’ “回” }%
\begin{tabular}{rl}\hline
default & \testtext\\\hline

\if0
umin10.tfm &
{\umin \testtext}\\\hline

ujis.tfm &
{\ujis \testtext}\\\hline
\fi

upjisr-h.tfm &
{\upjisr \testtext}\\\hline

\end{tabular}

\section{文字のサイズなど}
\xdef\testtext{%
\frame{:}~~\frame{“}\frame{回}\frame{:}\frame{回}\frame{”}~~“回:回”}%

{\huge
\begin{tabular}{rl}
default & \testtext\\

\if0
umin10.tfm &
{\umin \testtext}\\

ujis.tfm &
{\ujis \testtext}\\
\fi

upjisr-h.tfm &
{\upjisr \testtext}\\

\end{tabular}
}

\section{括弧類}
\subsection{JIS X 0208にあるもの}
‘回’ “回” (回) 〔回〕 [回] {回} 〈回〉 《回》 「回」 『回』 【回】

{\gt
‘回’ “回” (回) 〔回〕 [回] {回} 〈回〉 《回》 「回」 『回』 【回】
}

\subsection{JIS X 0213で追加されたもの}
⦅回⦆ 〘回〙 〖回〗 〝回〟

{\gt
⦅回⦆ 〘回〙 〖回〗 〝回〟
}

{\gt 〖回〗} は、KozGoProVI-Medium.otfならうまくいくはず。
\end{document}
