\documentclass{ujbook}
\usepackage{color}

\begin{document}
\chapter{青空文庫のサンプル}
\section{芥川龍之介「るしへる」}

破提宇子と云う天主教を弁難した書物のある事は、知っている人も少くあるまい。これは、元和六年、加賀の禅僧巴毗弇なるものの著した書物である。巴毗弇は当初南蛮寺に住した天主教徒であったが、その後何かの事情から、DS 如来を捨てて仏門に帰依する事になった。書中に云っている所から推すと、彼は老儒の学にも造詣のある、一かどの才子だったらしい。

破提宇子の流布本は、華頂山文庫の蔵本を、明治戊辰の頃、杞憂道人鵜飼徹定の序文と共に、出版したものである。が、そのほかにも異本がない訳ではない。現に予が所蔵の古写本の如きは、流布本と内容を異にする個所が多少ある。

中でも同書の第三段は、悪魔の起源を論じた一章であるが、流布本のそれに比して、予の蔵本では内容が遥に多い。巴毗弇自身の目撃した悪魔の記事が、あの辛辣な弁難攻撃の間に態々引証されてあるからである。この記事が流布本に載せられていない理由は、恐らくその余りに荒唐無稽に類する所から、こう云う破邪顕正を標榜する書物の性質上、故意の脱漏を利としたからでもあろうか。

予は以下にこの異本第三段を紹介して、聊巴毗弇の前に姿を現した、日本の Diabolus を一瞥しようと思う。なお巴毗弇に関して、詳細を知りたい人は、新村博士の巴毗弇に関する論文を一読するが好い。


\section{樋口一葉「大つごもり」}

お母樣御機嫌よう好い新年をお迎へなされませ、左樣ならば參りますと、暇乞わざと恭しく、お峰下駄を直せ、お玄關からお歸りではないお出かけだぞとづぶ\textcolor{blue}{〳〵}\footnote{青字はうまく組めない。}しく大手を振りて、行先は何處、父が涙は一夜の騷ぎに夢とやならん、持つまじきは放蕩息子、持つまじきは放蕩を仕立る繼母ぞかし。鹽花こそふらね跡は一先掃き出して、若旦那退散のよろこび、金は惜しけれど見る目も憎ければ家に居らぬは上々なり、何うすれば彼のやうに圖太くなられるか、あの子を生んだ母さんの顏が見たい、と御新造例に依つて毒舌をみがきぬ。お峰は此出來事も何として耳に入るべき、犯したる罪の恐ろしさに、我れか、人か、先刻の仕業はと今更夢路を辿りて、おもへば此事あらはれずして濟むべきや、萬が中なる一枚とても數ふれば目の前なるを、願ひの額に相應の員數手近の處になくなりしとあらば、我れにしても疑ひは何處に向くべき、調べられなば何とせん、何といはん、言ひ拔けんは罪深し、白状せば伯父が上にもかゝる、我罪は覺悟の上なれど物堅き伯父樣にまで濡れ衣を着せて、干されぬは貧乏のならひ、かゝる事もするものと人の言ひはせぬか、悲しや何としたらよかろ、伯父樣に疵のつかぬやう、我身が頓死する法は無きかと目は御新造が起居にしたがひて、心はかけ硯のもとにさまよひぬ。


\section{幸田露伴「雲のいろ\textcolor{blue}{〳〵}――卿雲」}

景雲といひ、卿雲といひ、慶雲といへる、しかと指し定められたる雲にはあらざるべし。卿雲爛たり糺縵〻たり、といへる、煙にあらず雲にあらず紫を曳き光を流す、といへる、大人作矣、五色氤氳、といへる、金柯初めて繞繚、玉葉漸く氤氳、といへる、還つて九霄に入りて沆\textcolor{red}{瀣}\footnote{赤字は第四水準の文字。}を成し、夕嵐生ずる處鶴松に歸る、といへる詩の句などによりて見れば、歸するところは美しき雲といふまでなり。一年の中に幾度か爛たる雲の見えざらん。若しまた餘りに美しき眼なれぬ雲などの出でたらんは、氣中のさまの常ならぬよりなるべければ、却つて悦ぶべからざるに似たり。五色の雲など何にせん、天は青きがめでたく、雲は白きこそ優しけれ。八雲立つの神の御歌を解きて、その時立ちし雲は天地のみたまの顯はせりし吉瑞にて、いともくしびなる雲なりけむなど橘の守部が云へるは、當れりや否や、知らず。くしびなる雲とは如何なる雲ぞや、問はまほし。八雲立ちといひたまはで、八雲立つと言い切り玉へるも彼の奇しき瑞雲に驚かせ給へる語勢なりなどいへる、ことに奇しき言なり。崇神紀の歌に、八雲立つ出雲梟師が云〻と歌へるも、八雲たちとは云はで八雲立つといひたるなれば、驚きたる語勢なりといふべきか、いと奇しき言なり。


\section{國木田獨歩「あの時分」}

「鸚鵡をくださいって」と、かごを取って去ってしまいました。この四郎さんは私と仲よしで、近いうちに裏の田んぼで雁をつる約束がしてあったのです、ところがその晩、おッ母アと樋口は某坂の町に買い物があるとて出てゆき、政法の二人は校堂でやる生徒仲間の演説会にゆき、木村は祈禱会にゆき、家に残ったのは、下女代わりに来ている親類の娘と、四郎と私だけで、すこぶるさびしくなりましたから、雁つりの実行に取りかかりました。


\section{森鷗外「百物語」}

玄関に上がる時に見ると、上がってすぐ突き当る三畳には、男が二人立って何か忙がしそうに咡き合っていた。「どうしやがったのだなあ」「それだからおいらが蠟燭は舟で来る人なんぞに持せて来ては行けないと云ったのだ。差当り燭台に立ててあるのしきゃないのだから」と云うような事を言っている。楽屋の方の世話も焼いている人達であろう。二人は僕の立っているのには構わずに、奥へ這入ってしまう。入り替って、一人の男が覗いて見て、黙って又引っ込んでしまう。

\end{document}

青空文庫作成ファイル:
このファイルは、インターネットの図書館、青空文庫(http://www.aozora.gr.jp/)で作られました。入力、校正、制作にあたったのは、ボランティアの皆さんです。
